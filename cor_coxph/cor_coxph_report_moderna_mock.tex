% Options for packages loaded elsewhere
\PassOptionsToPackage{unicode}{hyperref}
\PassOptionsToPackage{hyphens}{url}
\PassOptionsToPackage{dvipsnames,svgnames*,x11names*}{xcolor}
%
\documentclass[
]{article}
\usepackage{lmodern}
\usepackage{amssymb,amsmath}
\usepackage{ifxetex,ifluatex}
\ifnum 0\ifxetex 1\fi\ifluatex 1\fi=0 % if pdftex
  \usepackage[T1]{fontenc}
  \usepackage[utf8]{inputenc}
  \usepackage{textcomp} % provide euro and other symbols
\else % if luatex or xetex
  \usepackage{unicode-math}
  \defaultfontfeatures{Scale=MatchLowercase}
  \defaultfontfeatures[\rmfamily]{Ligatures=TeX,Scale=1}
\fi
% Use upquote if available, for straight quotes in verbatim environments
\IfFileExists{upquote.sty}{\usepackage{upquote}}{}
\IfFileExists{microtype.sty}{% use microtype if available
  \usepackage[]{microtype}
  \UseMicrotypeSet[protrusion]{basicmath} % disable protrusion for tt fonts
}{}
\makeatletter
\@ifundefined{KOMAClassName}{% if non-KOMA class
  \IfFileExists{parskip.sty}{%
    \usepackage{parskip}
  }{% else
    \setlength{\parindent}{0pt}
    \setlength{\parskip}{6pt plus 2pt minus 1pt}}
}{% if KOMA class
  \KOMAoptions{parskip=half}}
\makeatother
\usepackage{xcolor}
\IfFileExists{xurl.sty}{\usepackage{xurl}}{} % add URL line breaks if available
\IfFileExists{bookmark.sty}{\usepackage{bookmark}}{\usepackage{hyperref}}
\hypersetup{
  pdftitle={COVID correlates analysis report CoR},
  colorlinks=true,
  linkcolor=blue,
  filecolor=Maroon,
  citecolor=Blue,
  urlcolor=Blue,
  pdfcreator={LaTeX via pandoc}}
\urlstyle{same} % disable monospaced font for URLs
\usepackage[margin=1in]{geometry}
\usepackage{longtable,booktabs}
% Correct order of tables after \paragraph or \subparagraph
\usepackage{etoolbox}
\makeatletter
\patchcmd\longtable{\par}{\if@noskipsec\mbox{}\fi\par}{}{}
\makeatother
% Allow footnotes in longtable head/foot
\IfFileExists{footnotehyper.sty}{\usepackage{footnotehyper}}{\usepackage{footnote}}
\makesavenoteenv{longtable}
\usepackage{graphicx}
\makeatletter
\def\maxwidth{\ifdim\Gin@nat@width>\linewidth\linewidth\else\Gin@nat@width\fi}
\def\maxheight{\ifdim\Gin@nat@height>\textheight\textheight\else\Gin@nat@height\fi}
\makeatother
% Scale images if necessary, so that they will not overflow the page
% margins by default, and it is still possible to overwrite the defaults
% using explicit options in \includegraphics[width, height, ...]{}
\setkeys{Gin}{width=\maxwidth,height=\maxheight,keepaspectratio}
% Set default figure placement to htbp
\makeatletter
\def\fps@figure{htbp}
\makeatother
\setlength{\emergencystretch}{3em} % prevent overfull lines
\providecommand{\tightlist}{%
  \setlength{\itemsep}{0pt}\setlength{\parskip}{0pt}}
\setcounter{secnumdepth}{5}
\usepackage{float}
\usepackage{caption}
\usepackage{subcaption}
\usepackage{graphicx}
\usepackage[]{natbib}
\bibliographystyle{plainnat}

\title{COVID correlates analysis report CoR}
\author{}
\date{\vspace{-2.5em}2021-06-15}

\begin{document}
\maketitle

\counterwithin{table}{section}
\counterwithin{figure}{section}

\newif\ifENSEMBLE

\hypertarget{cor-coxph-Day57}{%
\section{Day 57 Univariate CoR: Cox Models of Risk}\label{cor-coxph-Day57}}

The main regression model is the Cox proportional hazards model. All plots are made with Cox models fit unless specified otherwise.

\hypertarget{hazard-ratios}{%
\subsection{Hazard ratios}\label{hazard-ratios}}

\begin{table}[H]
\caption{Inference for Day 57 antibody marker covariate-adjusted correlates of risk of COVID in the vaccine group: Hazard ratios per 10-fold increment in the marker*}
\begin{center}
    \input{output/D57/CoR_univariable_svycoxph_pretty_MockCOVE}\\
\end{center}
*Baseline covariates adjusted for: baseline risk score, at risk or not, community of color or not. Maximum failure event time \input{output/D57/timepoints_cum_risk_MockCOVE} days.\\
**No. at-risk = number of per-protocol baseline negative vaccine recipients at-risk for COVID; no. cases = number of this cohort with an observed COVID endpoints starting 7 days post Day 57 visit.\\
***q-value and FWER (family-wide error rate) are computed over the set of p-values both for quantitative markers and categorical markers (excluding ID80) using the Westfall and Young permutation method (\protect\input{output/D57/permutation_replicates_MockCOVE} replicates). 

    %\label{tab:CoR_univariable_svycoxph_pretty_MockCOVE}
\end{table}

\begin{table}[H]
\caption{Inference for Day 57 antibody marker covariate-adjusted correlates of risk of COVID in the vaccine group: Hazard ratios for Middle vs. Upper tertile vs. Lower tertile*}
\begin{center}
\setlength{\tabcolsep}{.5ex}
\input{output/D57/CoR_univariable_svycoxph_cat_pretty_MockCOVE}\\
\end{center}
*Baseline covariates adjusted for: baseline risk score, at risk or not, community of color or not. 
Maximum failure event time \protect\input{output/D57/timepoints_cum_risk_MockCOVE} days.
Cutpoints:
%Day 57 cutpoints:
\input{output/D57/cutpoints_Day57bindSpike_MockCOVE},  
\input{output/D57/cutpoints_Day57bindRBD_MockCOVE},  
\input{output/D57/cutpoints_Day57pseudoneutid50_MockCOVE},
\input{output/D57/cutpoints_Day57pseudoneutid80_MockCOVE},
all on the log10 scale.
%fold-rise cutpoints:
%\input{output/D57/cutpoints_Delta57overBbindSpike_MockCOVE},  
%\input{output/D57/cutpoints_Delta57overBbindRBD_MockCOVE},  
%\input{output/D57/cutpoints_Delta57overBpseudoneutid50_MockCOVE},  
%\input{output/D57/cutpoints_Delta57overBpseudoneutid80_MockCOVE}.  
\\
**No. at-risk = number of per-protocol baseline negative vaccine recipients at-risk for COVID at 7 days post Day 57 visit; no. cases = number of this cohort with an observed COVID endpoints.\\
***Generalized Wald-test p-value of the null hypothesis that the hazard rate is constant across the Lower, Middle, and Upper tertile groups.\\
$\dagger$ q-value and FWER (family-wide error rate) are computed over the set of p-values both for quantitative markers and categorical markers (excluding ID80) using the Westfall and Young permutation method (\protect\input{output/D57/permutation_replicates_MockCOVE} replicates). 

    %\label{tab:CoR_univariable_svycoxph_cat_pretty_MockCOVE}
\end{table}

\begin{figure}[H]
    \includegraphics[width=1\textwidth]{output/D57/hr_forest_bindSpike_MockCOVE}
    \includegraphics[width=1\textwidth]{output/D57/hr_forest_bindRBD_MockCOVE}
    \includegraphics[width=1\textwidth]{output/D57/hr_forest_pseudoneutid50_MockCOVE}
    \includegraphics[width=1\textwidth]{output/D57/hr_forest_pseudoneutid80_MockCOVE}
    \caption{Forest plots of hazard ratios per 10-fold increase in the marker among baseline negative vaccine recipients and subgroups with 95\% point-wise confidence intervals.}
\end{figure}
\clearpage

\begin{figure}[H]
    \centering
    \includegraphics[width=1\textwidth]{output/D57/hr_forest_marginal_bindSpike_MockCOVE}
    \caption{Forest plots of hazard ratios per 10-fold increase in the Day 57 binding Ab to spike markers among baseline negative vaccine recipients (top row) and eight subpopulations (row 2-9) with 95\% point-wise confidence intervals.}
\end{figure}

\begin{figure}[H]
    \centering
    \includegraphics[width=1\textwidth]{output/D57/hr_forest_marginal_bindRBD_MockCOVE}
    \caption{Forest plots of hazard ratios per 10-fold increase in the  Day 57 binding Ab to RBD markers among baseline negative vaccine recipients (top row) and eight subpopulations (row 2-9) with 95\% point-wise confidence intervals.}
\end{figure}

\ifENSEMBLE
\else

\begin{figure}[H]
    \centering
    \includegraphics[width=1\textwidth]{output/D57/hr_forest_marginal_pseudoneutid50_MockCOVE}
    \caption{Forest plots of hazard ratios per 10-fold increase in the  Day 57 pseudo neut ID50 markers among baseline negative vaccine recipients (top row) and eight subpopulations (row 2-9) with 95\% point-wise confidence intervals.}
\end{figure}

\begin{figure}[H]
    \centering
    \includegraphics[width=1\textwidth]{output/D57/hr_forest_marginal_pseudoneutid80_MockCOVE}
    \caption{Forest plots of hazard ratios per 10-fold increase in the  Day 57 pseudo neut ID80 markers among baseline negative vaccine recipients (top row) and eight subpopulations (row 2-9) with 95\% point-wise confidence intervals.}
\end{figure}

\fi

\clearpage

\newif\ifShowCountry

\ifShowCountry

\begin{figure}[H]
    \centering
    \includegraphics[width=1\textwidth]{output/D57/hr_forest_countries_bindSpike_MockCOVE}
    \caption{Forest plots of hazard ratios per 10-fold increase in the Day 57 binding Ab to spike markers among baseline negative vaccine recipients (top row) and eight subpopulations (row 2-9) with 95\% point-wise confidence intervals.}
\end{figure}

\begin{figure}[H]
    \centering
    \includegraphics[width=1\textwidth]{output/D57/hr_forest_countries_bindRBD_MockCOVE}
    \caption{Forest plots of hazard ratios per 10-fold increase in the  Day 57 binding Ab to RBD markers among baseline negative vaccine recipients (top row) and eight subpopulations (row 2-9) with 95\% point-wise confidence intervals.}
\end{figure}

\fi

\hypertarget{marginalized-risk-and-controlled-vaccine-efficacy-plots}{%
\subsection{Marginalized risk and controlled vaccine efficacy plots}\label{marginalized-risk-and-controlled-vaccine-efficacy-plots}}

\begin{center}
\begingroup
\renewcommand{\arraystretch}{2} % Default value: 1
\begin{table}[H] \centering
\doublespacing
\singlespacing
\begin{tabular}{lcccccc} \hline \hline
 & \multicolumn{2}{c}{marginalized risk} & \multicolumn{2}{c}{controlled risk} & & \\ 
 & \multicolumn{2}{c}{ratio $RR_M(0,1)$} & \multicolumn{2}{c}{ratio $RR_C(0,1)^1$} & \multicolumn{2}{c}{e(0,1)$^2$} \\ 
Trial  & Point Est. & 95\% CI  &    Point Est. & 95\% CI  &   Point Est.  & 95\% CI UL \\ \hline
\input{input/output/D57/marginalized_risks_cat_MockCOVE}  \\ \hline \hline
\end{tabular}
\newline
\caption{Analysis of Day 57 markers (upper vs. lower tertile) as a CoR and a controlled risk CoP.}
\doublespacing
\noindent $^1$Conservative (upper bound) estimate assuming unmeasured confounding at level $RR_{UD}(0,1)=RR_{EU}(0,1) = 2$ and thus $B(0,1)=4/3$. \newline
\noindent $^2$E-values are computed for upper tertile $s=1$ vs. lower tertile $s=0$ biomarker subgroups after controlling for baseline risk score, at risk or not, community of color or not; UL = upper limit.
\label{table1}
\end{table}
\endgroup
\end{center}

\begin{figure}[H]
    \centering
    \includegraphics[width=1\textwidth]{output/D57/marginalized_risks_cat_MockCOVE}
    \caption{Marginalized cumulative incidence rate curves for trichotomized Day 57 markers among baseline negative vaccine recipients. The gray line is the overall cumulative incidence rate curve in the placebo arm.}
\end{figure}

\begin{figure}[H]
    \centering
    \includegraphics[width=1\textwidth]{output/D57/marginalized_risks_eq_MockCOVE}
    \caption{Marginalized cumulative risk by Day \protect\input{output/D57/timepoints_cum_risk_MockCOVE} as functions of Day 57 markers (=s) among baseline negative vaccine recipients with 95\% bootstrap point-wise confidence bands (\protect\input{output/D57/bootstrap_replicates_MockCOVE} replicates). The horizontal lines indicate the overall cumulative risk of the placebo and vaccine arms by Day \protect\input{output/D57/timepoints_cum_risk_MockCOVE} and its 95\% point-wise confidence interval. Histograms of the immunological markers in the vaccine arm are overlaid. lod = 0.3, 1.6, 2.4, 15 for bAb Spike, bAb RBD, PsV nAb ID50, PsV nAb ID80, respectively.}
    
\end{figure}

\begin{figure}[H]
    \centering
    \includegraphics[width=1\textwidth]{output/D57/controlled_ve_curves_eq_MockCOVE}
    \caption{Controlled VE with sensitivity analysis as functions of Day 57 markers (=s) among baseline negative vaccine recipients with 95\% bootstrap point-wise confidence bands (\protect\input{output/D57/bootstrap_replicates_MockCOVE} replicates). Histograms of the immunological markers in the vaccine arm are overlaid. lod = 0.3, 1.6, 2.4, 15 for bAb Spike, bAb RBD, PsV nAb ID50, PsV nAb ID80, respectively.}
\end{figure}

\begin{figure}[H]
    \centering
    \includegraphics[width=1\textwidth]{output/D57/marginalized_risks_geq_woplacebo_MockCOVE}
    \caption{Marginalized cumulative risk by Day \protect\input{output/D57/timepoints_cum_risk_MockCOVE} as functions of Day 57 markers above a threshold ($\geq s$) among baseline negative vaccine recipients with 95\% bootstrap point-wise confidence bands (at least 5 cases are required, \protect\input{output/D57/bootstrap_replicates_MockCOVE} replicates). The horizontal lines indicate the overall cumulative risk of the vaccine arm by Day \protect\input{output/D57/timepoints_cum_risk_MockCOVE} and its 95\% point-wise confidence interval. Histograms of the immunological markers in the vaccine arm are overlaid. lod = 0.3, 1.6, 2.4, 15 for bAb Spike, bAb RBD, PsV nAb ID50, PsV nAb ID80, respectively.}
\end{figure}

\begin{figure}[H]
    \centering
    \includegraphics[width=1\textwidth]{output/D57/controlled_ve_curves_geq_MockCOVE}
    \caption{Controlled VE as functions of Day 57 markers (>=s) among baseline negative vaccine recipients with 95\% bootstrap point-wise confidence bands (\protect\input{output/D57/bootstrap_replicates_MockCOVE} replicates). Histograms of the immunological markers in the vaccine arm are overlaid. lod = 0.3, 1.6, 2.4, 15 for bAb Spike, bAb RBD, PsV nAb ID50, PsV nAb ID80, respectively.}
\end{figure}

\clearpage

\hypertarget{cor-coxph-Day29}{%
\section{Day 29 Univariate CoR: Cox Models of Risk}\label{cor-coxph-Day29}}

The main regression model is the Cox proportional hazards model. All plots are made with Cox models fit unless specified otherwise.

\hypertarget{hazard-ratios-1}{%
\subsection{Hazard ratios}\label{hazard-ratios-1}}

\begin{table}[H]
\caption{Inference for Day 29 antibody marker covariate-adjusted correlates of risk of COVID in the vaccine group: Hazard ratios per 10-fold increment in the marker*}
\begin{center}
    \input{output/D29/CoR_univariable_svycoxph_pretty_MockCOVE}\\
\end{center}
*Baseline covariates adjusted for: baseline risk score, at risk or not, community of color or not. Maximum failure event time \input{output/D29/timepoints_cum_risk_MockCOVE} days.\\
**No. at-risk = number of per-protocol baseline negative vaccine recipients at-risk for COVID; no. cases = number of this cohort with an observed COVID endpoints starting 7 days post Day 29 visit.\\
***q-value and FWER (family-wide error rate) are computed over the set of p-values both for quantitative markers and categorical markers (excluding ID80) using the Westfall and Young permutation method (\protect\input{output/D29/permutation_replicates_MockCOVE} replicates). 

    %\label{tab:CoR_univariable_svycoxph_pretty_MockCOVE}
\end{table}

\begin{table}[H]
\caption{Inference for Day 29 antibody marker covariate-adjusted correlates of risk of COVID in the vaccine group: Hazard ratios for Middle vs. Upper tertile vs. Lower tertile*}
\begin{center}
\setlength{\tabcolsep}{.5ex}
\input{output/D29/CoR_univariable_svycoxph_cat_pretty_MockCOVE}\\
\end{center}
*Baseline covariates adjusted for: baseline risk score, at risk or not, community of color or not. 
Maximum failure event time \protect\input{output/D29/timepoints_cum_risk_MockCOVE} days.
Cutpoints:
%Day 29 cutpoints:
\input{output/D29/cutpoints_Day29bindSpike_MockCOVE},  
\input{output/D29/cutpoints_Day29bindRBD_MockCOVE},  
\input{output/D29/cutpoints_Day29pseudoneutid50_MockCOVE},
\input{output/D29/cutpoints_Day29pseudoneutid80_MockCOVE},
all on the log10 scale.
%fold-rise cutpoints:
%\input{output/D29/cutpoints_Delta29overBbindSpike_MockCOVE},  
%\input{output/D29/cutpoints_Delta29overBbindRBD_MockCOVE},  
%\input{output/D29/cutpoints_Delta29overBpseudoneutid50_MockCOVE},  
%\input{output/D29/cutpoints_Delta29overBpseudoneutid80_MockCOVE}.  
\\
**No. at-risk = number of per-protocol baseline negative vaccine recipients at-risk for COVID at 7 days post Day 29 visit; no. cases = number of this cohort with an observed COVID endpoints.\\
***Generalized Wald-test p-value of the null hypothesis that the hazard rate is constant across the Lower, Middle, and Upper tertile groups.\\
$\dagger$ q-value and FWER (family-wide error rate) are computed over the set of p-values both for quantitative markers and categorical markers (excluding ID80) using the Westfall and Young permutation method (\protect\input{output/D29/permutation_replicates_MockCOVE} replicates). 

    %\label{tab:CoR_univariable_svycoxph_cat_pretty_MockCOVE}
\end{table}

\begin{figure}[H]
    \includegraphics[width=1\textwidth]{output/D29/hr_forest_bindSpike_MockCOVE}
    \includegraphics[width=1\textwidth]{output/D29/hr_forest_bindRBD_MockCOVE}
    \includegraphics[width=1\textwidth]{output/D29/hr_forest_pseudoneutid50_MockCOVE}
    \includegraphics[width=1\textwidth]{output/D29/hr_forest_pseudoneutid80_MockCOVE}
    \caption{Forest plots of hazard ratios per 10-fold increase in the marker among baseline negative vaccine recipients and subgroups with 95\% point-wise confidence intervals.}
\end{figure}
\clearpage

\begin{figure}[H]
    \centering
    \includegraphics[width=1\textwidth]{output/D29/hr_forest_marginal_bindSpike_MockCOVE}
    \caption{Forest plots of hazard ratios per 10-fold increase in the Day 29 binding Ab to spike markers among baseline negative vaccine recipients (top row) and eight subpopulations (row 2-9) with 95\% point-wise confidence intervals.}
\end{figure}

\begin{figure}[H]
    \centering
    \includegraphics[width=1\textwidth]{output/D29/hr_forest_marginal_bindRBD_MockCOVE}
    \caption{Forest plots of hazard ratios per 10-fold increase in the  Day 29 binding Ab to RBD markers among baseline negative vaccine recipients (top row) and eight subpopulations (row 2-9) with 95\% point-wise confidence intervals.}
\end{figure}

\ifENSEMBLE
\else

\begin{figure}[H]
    \centering
    \includegraphics[width=1\textwidth]{output/D29/hr_forest_marginal_pseudoneutid50_MockCOVE}
    \caption{Forest plots of hazard ratios per 10-fold increase in the  Day 29 pseudo neut ID50 markers among baseline negative vaccine recipients (top row) and eight subpopulations (row 2-9) with 95\% point-wise confidence intervals.}
\end{figure}

\begin{figure}[H]
    \centering
    \includegraphics[width=1\textwidth]{output/D29/hr_forest_marginal_pseudoneutid80_MockCOVE}
    \caption{Forest plots of hazard ratios per 10-fold increase in the  Day 29 pseudo neut ID80 markers among baseline negative vaccine recipients (top row) and eight subpopulations (row 2-9) with 95\% point-wise confidence intervals.}
\end{figure}

\fi

\clearpage

\newif\ifShowCountry

\ifShowCountry

\begin{figure}[H]
    \centering
    \includegraphics[width=1\textwidth]{output/D29/hr_forest_countries_bindSpike_MockCOVE}
    \caption{Forest plots of hazard ratios per 10-fold increase in the Day 29 binding Ab to spike markers among baseline negative vaccine recipients (top row) and eight subpopulations (row 2-9) with 95\% point-wise confidence intervals.}
\end{figure}

\begin{figure}[H]
    \centering
    \includegraphics[width=1\textwidth]{output/D29/hr_forest_countries_bindRBD_MockCOVE}
    \caption{Forest plots of hazard ratios per 10-fold increase in the  Day 29 binding Ab to RBD markers among baseline negative vaccine recipients (top row) and eight subpopulations (row 2-9) with 95\% point-wise confidence intervals.}
\end{figure}

\fi

\hypertarget{marginalized-risk-and-controlled-vaccine-efficacy-plots-1}{%
\subsection{Marginalized risk and controlled vaccine efficacy plots}\label{marginalized-risk-and-controlled-vaccine-efficacy-plots-1}}

\textbackslash begin\{center\}
\begingroup

\renewcommand{\arraystretch}{2}

\% Default value: 1

\begin{table}[H] \centering
\doublespacing
\singlespacing
\begin{tabular}{lcccccc} \hline \hline
 & \multicolumn{2}{c}{marginalized risk} & \multicolumn{2}{c}{controlled risk} & & \\ 
 & \multicolumn{2}{c}{ratio $RR_M(0,1)$} & \multicolumn{2}{c}{ratio $RR_C(0,1)^1$} & \multicolumn{2}{c}{e(0,1)$^2$} \\ 
Trial  & Point Est. & 95\% CI  &    Point Est. & 95\% CI  &   Point Est.  & 95\% CI UL \\ \hline
\input{input/output/D29/marginalized_risks_cat_MockCOVE}  \\ \hline \hline
\end{tabular}
\newline
\caption{Analysis of Day 29 markers (upper vs. lower tertile) as a CoR and a controlled risk CoP.}
\doublespacing
\noindent $^1$Conservative (upper bound) estimate assuming unmeasured confounding at level $RR_{UD}(0,1)=RR_{EU}(0,1) = 2$ and thus $B(0,1)=4/3$. \newline
\noindent $^2$E-values are computed for upper tertile $s=1$ vs. lower tertile $s=0$ biomarker subgroups after controlling for baseline risk score, at risk or not, community of color or not; UL = upper limit.
\label{table1}
\end{table}
\endgroup

\textbackslash end\{center\}

\begin{figure}[H]
    \centering
    \includegraphics[width=1\textwidth]{output/D29/marginalized_risks_cat_MockCOVE}
    \caption{Marginalized cumulative incidence rate curves for trichotomized Day 29 markers among baseline negative vaccine recipients. The gray line is the overall cumulative incidence rate curve in the placebo arm.}
\end{figure}

\begin{figure}[H]
    \centering
    \includegraphics[width=1\textwidth]{output/D29/marginalized_risks_eq_MockCOVE}
    \caption{Marginalized cumulative risk by Day \protect\input{output/D29/timepoints_cum_risk_MockCOVE} as functions of Day 29 markers (=s) among baseline negative vaccine recipients with 95\% bootstrap point-wise confidence bands (\protect\input{output/D29/bootstrap_replicates_MockCOVE} replicates). The horizontal lines indicate the overall cumulative risk of the placebo and vaccine arms by Day \protect\input{output/D29/timepoints_cum_risk_MockCOVE} and its 95\% point-wise confidence interval. Histograms of the immunological markers in the vaccine arm are overlaid. lod = 0.3, 1.6, 2.4, 15 for bAb Spike, bAb RBD, PsV nAb ID50, PsV nAb ID80, respectively.}
    
\end{figure}

\begin{figure}[H]
    \centering
    \includegraphics[width=1\textwidth]{output/D29/controlled_ve_curves_eq_MockCOVE}
    \caption{Controlled VE with sensitivity analysis as functions of Day 29 markers (=s) among baseline negative vaccine recipients with 95\% bootstrap point-wise confidence bands (\protect\input{output/D29/bootstrap_replicates_MockCOVE} replicates). Histograms of the immunological markers in the vaccine arm are overlaid. lod = 0.3, 1.6, 2.4, 15 for bAb Spike, bAb RBD, PsV nAb ID50, PsV nAb ID80, respectively.}
\end{figure}

\begin{figure}[H]
    \centering
    \includegraphics[width=1\textwidth]{output/D29/marginalized_risks_geq_woplacebo_MockCOVE}
    \caption{Marginalized cumulative risk by Day \protect\input{output/D29/timepoints_cum_risk_MockCOVE} as functions of Day 29 markers above a threshold ($\geq s$) among baseline negative vaccine recipients with 95\% bootstrap point-wise confidence bands (at least 5 cases are required, \protect\input{output/D29/bootstrap_replicates_MockCOVE} replicates). The horizontal lines indicate the overall cumulative risk of the vaccine arm by Day \protect\input{output/D29/timepoints_cum_risk_MockCOVE} and its 95\% point-wise confidence interval. Histograms of the immunological markers in the vaccine arm are overlaid. lod = 0.3, 1.6, 2.4, 15 for bAb Spike, bAb RBD, PsV nAb ID50, PsV nAb ID80, respectively.}
\end{figure}

\begin{figure}[H]
    \centering
    \includegraphics[width=1\textwidth]{output/D29/controlled_ve_curves_geq_MockCOVE}
    \caption{Controlled VE as functions of Day 29 markers (>=s) among baseline negative vaccine recipients with 95\% bootstrap point-wise confidence bands (\protect\input{output/D29/bootstrap_replicates_MockCOVE} replicates). Histograms of the immunological markers in the vaccine arm are overlaid. lod = 0.3, 1.6, 2.4, 15 for bAb Spike, bAb RBD, PsV nAb ID50, PsV nAb ID80, respectively.}
\end{figure}

  \bibliography{ref.bib}

\end{document}
